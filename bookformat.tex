\documentclass[12pt,a4paper,UTF8,AutoFakeBold]{ctexbook} %注意AutoFackBold打开用来加粗中文字体
\usepackage[cache=false]{minted}  % 代码高亮 必须在编译命令后面加上x -shell-escape; 在options->Config..->command
\usepackage{listings} %代码高亮的另一种实现
\usepackage{xcolor}
\usepackage{xeCJK}
\usepackage{graphicx}
\usepackage{amsmath}
\usepackage{ctexcap}
\usepackage{fancyhdr}
\usepackage{caption2}
\usepackage[colorlinks=true, linkcolor=blue]{hyperref} %目录和交叉引用超链接

%-------------------------------浮动对象间距------------------------------------------
%页面的顶部或底部浮体与文本之间http://bbs.ctex.org/forum.php?mod=viewthread&tid=70899
\setlength{\textfloatsep} {-10pt} 
%页面顶部或者底部浮体的距离 http://www.ctex.org/documents/latex/graphics/node72.html
\setlength{\floatsep}{0pt} 
%页面中浮体图形与上下方距离http://www.ctex.org/documents/latex/graphics/node72.html
\setlength{\intextsep}{10pt} 

%-------------------------------图表设置----------------------------------------------
\renewcommand{\captionfont}{\songti\zihao{5}} %图标题5号
\captionsetup{format=hang, aboveskip=0pt, belowskip=0pt}
%插入图片的命令
\newcommand{\myfig}[4]{%#1标题 #2label #3宽度 #4img_path
		\begin{figure}[!ht]
		\begin{center}
			\setlength{\belowcaptionskip}{10pt}
			\includegraphics[width=#3\textwidth]{#4}
			\caption{#1}\label{#2}
		\end{center}
	\end{figure}
}

%插入表格的命令
%#1标题 #2label #3col_format #4 table_content
\newenvironment{mytable}[3]{
	\begin{table}[!ht]
		\zihao{5}\centering
		\caption{#1}\label{#2}
		\begin{tabular}{#3}	
		}{
		\end{tabular}
	\end{table}
}

%举一表插入的例子OpenFlow Type table
\newcommand{\oftypetable}{
	\newcounter{rownum}
	\newcommand{\rown}{\stepcounter{rownum}\therownum}
	\begin{mytable}{OpenFLow包格式}{tab:oftype}{|c|c|c|}
		\hline
		序号&类型&描述\\ \hline
		\rown &  Packet-In   & 发送到控制器\\ \hline
		\rown &  Packet-Out  & 发到交换机 \\ \hline
		\rown &  Flow-Mod    & 修改流表项,增删改查,控制器发送到交换机的 \\ \hline
	\end{mytable}
}


%引用学习 http://www.latexstudio.net/archives/3297
%http://blog.sina.com.cn/s/blog_7101508c0100tgg4.html
%------------------------------页眉面脚-----------------------------------------
%页眉面脚 参见《latex2e完全学习手册》4.4版式 页眉与页脚
\pagestyle{fancy}
\fancyhead[EL,OL,ER,OR]{}
\fancyhead[EC]{西安电子科技大学}
\fancyhead[OC]{\leftmark}
\fancyfoot[C]{\thepage}


%文档
%CTEX小手册http://www.docin.com/p-57589811.html
%手册 http://www.ctex.org/documents/latex/graphics/node2.html
%http://www.docin.com/p-1227477687.html?docfrom=rrela

%-------------------------------全文字体、字号-------------------------------------
%字体设置参考http://blog.csdn.net/archielau/article/details/7960229
\setmainfont{Times New Roman}
\setCJKmainfont{SimSun}
%全文字号小四=12.1pt
\renewcommand{\normalsize}{\fontsize{12.1pt}{\baselineskip}\selectfont}

%-------------------------段间距、行间距、首行缩进----------------------------------
%段落设置 需要在每个独立的文件中包含的
\newcommand{\mylineskip}{\setlength{\baselineskip}{20pt}} %行间距20磅
\mylineskip
%LaTeX技巧459:LaTeX中常见段落格式的设定-字间距,行间距,段间距,缩进 http://blog.sina.com.cn/s/blog_5e16f1770100ns4r.html
%段间距设置段前后都是0磅 
\setlength{\parskip}{0pt}
%首行缩进两字符
\CTEXindent

%---------------------------------三级标题格式--------------------------------------
%标题设置
%ctexset 用法参见:http://www.docin.com/p-759953748.html
%一级标题居中排列,字体为黑体,字号为三号,行距为固定值20磅,段落间距为段前24磅,段后18磅
\CTEXsetup[beforeskip={24pt}, afterskip={18pt}, format={\huge\heiti\zihao{3}\centering\color{cyan}}]{chapter} %一级标题黑体三号

%二级标题不缩进,字体为宋体加粗,字号为小三号,行距为固定值20磅,段落间距为段前18磅,段后12磅;
\CTEXsetup[beforeskip={18pt},afterskip={12pt},format={\Large\bf\songti\zihao{-3}\flushleft\color{blue}}]{section}

%三级标题缩进2字符,字体为宋体,字号为四号加粗,行距为固定值20磅,段落间距为段前12磅,段后6磅;
\CTEXsetup[indent={2\ccwd},beforeskip={12pt},afterskip={6pt},format={\large\bf\songti\zihao{4}\color{olive}}]{subsection}

